
% Define Document Class
\documentclass[a4paper,twoside]{article}

% Import Packages
\usepackage{graphicx}
\usepackage[style=ieee]{biblatex} % Set the reference style to IEEE
\usepackage{xcolor}
\usepackage{hyperref}
\usepackage{titletoc}
\usepackage{fancyhdr}
\usepackage{blindtext}
\usepackage{titlesec}
\usepackage{tocloft}
\usepackage{listings}
\usepackage{array} % For customizing the table
\usepackage{booktabs} % For better quality horizontal lines
\usepackage{graphicx} % Package for including images
\usepackage{float}
\usepackage{subcaption}
\usepackage{biblatex}
\usepackage[printonlyused]{acronym}
\usepackage{color,soul}
\usepackage{amsmath}
\usepackage[parfill]{parskip}
\usepackage{textcomp}
\usepackage{siunitx}  % propper typesetting of numbers and units
\usepackage{multicol}  % multiple equations next to eachother
\usepackage{pdfpages}  % import PDFs
\usepackage{svg}


% Individualization
\newcommand{\titlename}{750N Engine Critical Design Review (CDR) Document}
\newcommand{\subtitlename}{Review of the Engine and Test Stand}
\newcommand{\filename}{BEARS\_Propulsion\_750N\_Engine\_CDR\_Document\_Version\_1\_0}
\newcommand{\projectname}{Propulsion Group}
\newcommand{\contactname}{Fabian Jonathan Krech}
\newcommand{\contactemail}{propulsion@bears-space.de}
\newcommand{\versionname}{1.0}
\newcommand{\issuedate}{\today}
\newcommand{\documenttype}{Spec}
\newcommand{\validfromdate}{\today}
\newcommand{\siunit}[0]{\,\text}


% Nomenclature Commands
\usepackage{amssymb}
\usepackage{nomencl}
\makenomenclature
\usepackage{etoolbox}
 \usepackage{siunitx}
%% This code creates the groups
% -----------------------------------------
\renewcommand\nomgroup[1]{%
  \item[\bfseries
  \ifstrequal{#1}{L}{Latin Symbols}{%
  \ifstrequal{#1}{G}{Greek Symbols}{%
  \ifstrequal{#1}{I}{Indices}{}}}%
]}
% -----------------------------------------

% Define margins
\usepackage[margin=1in]{geometry}

% Settings for Header
\fancypagestyle{plain}{
  \fancyhf{}
  \fancyhead[LO,RE]{\includegraphics[height=15pt]{logos/BEARS_Logo.png}} % Logo links
  \fancyhead[C]{\leftmark} % Aktuelle Section in der Mitte
  \fancyhead[LE,RO]{\thepage} % Seitenzahl rechts
  \renewcommand{\headrulewidth}{0.4pt}
  \fancyfoot[LO,RE]{\filename}
  \renewcommand{\footrulewidth}{0.4pt}
  \setlength{\headheight}{18.60004pt}
}

\fancypagestyle{noheader}{
  \fancyhf{}
  \fancyfoot[C]{\thepage} % Seitenzahl unten zentriert
  \renewcommand{\headrulewidth}{0pt}
}

% Code Listings Customization
\definecolor{codegreen}{rgb}{0,0.6,0}
\definecolor{codegray}{rgb}{0.5,0.5,0.5}
\definecolor{codepurple}{rgb}{0.58,0,0.82}
\definecolor{backcolour}{rgb}{0.95,0.95,0.92}

\lstdefinestyle{mystyle}{
    backgroundcolor=\color{backcolour},   
    commentstyle=\color{codegreen},
    keywordstyle=\color{magenta},
    numberstyle=\tiny\color{codegray},
    stringstyle=\color{codepurple},
    basicstyle=\ttfamily\footnotesize,
    breakatwhitespace=false,         
    breaklines=true,                 
    captionpos=b,                    
    keepspaces=true,                 
    numbers=left,                    
    numbersep=5pt,                  
    showspaces=false,                
    showstringspaces=false,
    showtabs=false,                  
    tabsize=2
}

\lstset{style=mystyle}

% Bibliography Settings
\addbibresource{bibliography.bib}

